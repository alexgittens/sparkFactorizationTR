We conclude our study of matrix factorizations at scale with the following take-away messages: 
\begin{itemize}
  \item{A range of important matrix factorization algorithms can be implemented in Spark: we have successfully applied NMF, PCA and CX to TB-sized da tasets. We have scaled the codes on 50, 100, 300, 500, and 1600 XC40 nodes. To the best of our knowledge, these are some of the largest scale \emph{scientific data analytics} workloads attempted with Spark.}
\item{Spark and C+MPI head-to-head comparisons of these methods have revealed a number of opportunities for improving Spark performance. The current end-to-end performance gap for our workloads is $2\times - 25\times$; and $10\times - 40\times$ without I/O. At scale, Spark performance overheads associated with scheduling, stragglers, result serialization and task deserialization dominate the runtime by an order of magnitude.}
\item{{In order for Spark to leverage existing, high-performance linear algebra libraries, it may be worthwhile to investigate better mechanisms for integrating and interfacing with MPI-based runtimes with Spark. The cost associated with copying data between the runtimes may not be prohibitive.}}
\item{Finally, efficient, parallel I/O is critical for Data Analytics at scale. HPC system architectures will need to be balanced to support data-intensive workloads.}
\end{itemize}
